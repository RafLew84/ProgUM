\chapter{Wyklad 9: Retrofit i integracja z API zewnetrznymi}

\begin{learningobjectives}
  \item Skonfigurujesz Retrofit do pracy z API i serializerami.
  \item Zastosujesz interceptory OkHttp oraz strategie retry.
  \item Zabezpieczysz klucze API i obsluzysz bledy sieciowe.
\end{learningobjectives}

\section{Mapa rozdzialu}
\begin{itemize}
  \item Retrofit setup i konwertery.
  \item OkHttp interceptors, logging, timeouts.
  \item Autoryzacja i zarzadzanie kluczami API.
  \item Error handling, retry, exponential backoff.
  \item Integracja z Repository i coroutines.
\end{itemize}

\section{Material zrodlowy}
\begin{itemize}
  \item PUM2/Wyk/Wyklad10.pdf.
  \item \texttt{PUM2/Listy/W10-Retrofit\_basics.ipynb}, \texttt{PUM2/Listy/W10-Header\_basics.ipynb}.
  \item square.github.io/retrofit.
\end{itemize}

\subsection{Konfiguracja Retrofit}
\begin{itemize}
  \item BaseUrl, converter, call adapter.
  \item Moshi vs Kotlinx serialization.
  \item Coroutine support i suspend functions.
\end{itemize}

\subsection{OkHttp i interceptory}
\begin{itemize}
  \item Logging, header i auth interceptors.
  \item Timeouts, caching, retry policy.
  \item Monitoring i metrics.
\end{itemize}

\subsection{Bezpieczenstwo i bledy}
\begin{itemize}
  \item Przechowywanie kluczy API i secrets management.
  \item Error mapping, sealed wynik operacji.
  \item Retry, exponential backoff, circuit breaker.
\end{itemize}

\subsection{Case study}
\begin{itemize}
  \item Integracja API produktow.
  \item Zadanie: obsluga bledow i fallback offline.
  \item Checklist: monitoring, logging, analytics.
\end{itemize}

\section{Elementy do rozbudowy}
\begin{itemize}
  \item Przyklad A: Retrofit service z interceptorem autoryzacji (TODO).
  \item Przyklad B: wrapper na wyniki zapytan (TODO).
  \item Zadania: implementacja retry z exponential backoff (TODO).
  \item Pytania: rola call adapter, jak chronic klucze API (TODO).
\end{itemize}

% TODO: uzupelnic narracje, przyklady kodu i cytowania na etapie draftu.
