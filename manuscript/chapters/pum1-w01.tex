\chapter{Wyklad 1: Podstawy jezyka Kotlin}

\begin{learningobjectives}
  \item Przypomnisz sobie podstawowe elementy skladni Kotlin,
  \item zrozumiesz roznice miedzy `val` i `var`,
  \item napiszesz proste funkcje i petle kontrolujace przeplyw programu.
\end{learningobjectives}

\section{Streszczenie rozdzialu}
Rozdzial rozwija material z `PUM1/Wyk/Wyklad1.pdf`, dodajac wiecej przykladow kodu i komentarzy. W trakcie redakcji nalezy uzupelnic pelne przyklady wraz z wynikami uruchomienia oraz sekcja \term{Najczestsze bledy} z praktyki laboratoryjnej.

\section{Przyklad: interaktywna sesja Kotlin REPL}
\begin{minted}{kotlin}
fun main() {
  val greeting = "Czesc"
  var counter = 0

  repeat(3) {
    counter += 1
    println("$greeting, to jest powtorzenie #$counter")
  }
}
\end{minted}

W docelowej wersji nalezy dodac opis dzialania kazdej linii oraz propozycje modyfikacji zadaniowych.

\section{Zadania do samodzielnego wykonania}
% TODO: uzupelnic o zadania na bazie Listy 1
\begin{enumerate}[label=Z\arabic*.]
  \item Przeksztalc powyzszy przyklad tak, aby wypisywal wyrazenia arytmetyczne oraz ich wynik.
  \item Zaimplementuj mini quiz dzialajacy w konsoli wykorzystujacy instrukcje `when` i `try/catch`.
\end{enumerate}

