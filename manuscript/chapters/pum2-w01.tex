\chapter{Wyklad 1: Wprowadzenie do wielowatkowosci i coroutines}

\begin{learningobjectives}
  \item Przypomnisz podstawy wielowatkowosci i zagrozen w Androidzie.
  \item Poznasz fundamenty Kotlin Coroutines i korzysci structured concurrency.
  \item Zaprojektujesz proste przypadki uzycia coroutines w aplikacji mobilnej.
\end{learningobjectives}

\section{Mapa rozdzialu}
\begin{itemize}
  \item Wielowatkowosc w Android: UI thread, ANR, worker threads.
  \item Kotlin Coroutines: scope, context, job, dispatcher.
  \item Uruchamianie i anulowanie coroutines.
  \item Buildery launch, async, withContext.
  \item Integracja z ViewModel i Compose.
\end{itemize}

\section{Material zrodlowy}
\begin{itemize}
  \item PUM2/Wyk/Wyklad2.pdf.
  \item \texttt{PUM2/Listy/W2-Podstawy\_Coroutine.ipynb}.
  \item kotlinlang.org/docs/coroutines-overview.html.
\end{itemize}

\subsection{Dlaczego coroutines}
\begin{itemize}
  \item Problemy tradycyjnych watkow i callback hell.
  \item Koncepcje suspending functions, continuations.
  \item Porownanie z RxJava i handlerami.
\end{itemize}

\subsection{Podstawy API}
\begin{itemize}
  \item CoroutineScope, Job, Dispatcher.
  \item launch, async, withContext.
  \item Anulowanie i obsluga bledow.
\end{itemize}

\subsection{Integracja z aplikacja}
\begin{itemize}
  \item ViewModelScope, lifecycleScope.
  \item rememberCoroutineScope i Compose.
  \item Najlepsze praktyki dla UI thread.
\end{itemize}

\subsection{Case study}
\begin{itemize}
  \item Zadanie: pobieranie danych w tle i aktualizacja UI.
  \item Analiza bledow ANR i timeouts.
  \item Checklist: safe coroutine usage w projekcie.
\end{itemize}

\section{Elementy do rozbudowy}
\begin{itemize}
  \item Przyklad A: pobranie danych w coroutine z aktualizacja UI (TODO).
  \item Przyklad B: implementacja timeout z withTimeout (TODO).
  \item Zadania: refaktoryzacja zadan laboratoryjnych na coroutines (TODO).
  \item Pytania: roznica launch i async, rola dispatcherow (TODO).
\end{itemize}

% TODO: uzupelnic narracje, przyklady kodu i cytowania na etapie draftu.
