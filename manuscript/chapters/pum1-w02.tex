\chapter{Wyklad 2: Funkcje i funkcje wyzszego rzedu}

\begin{learningobjectives}
  \item Przypomnisz skladnie deklaracji funkcji i parametrow w Kotlinie.
  \item Zidentyfikujesz roznice miedzy funkcjami zwyklymi, wyrazeniowymi i lambdami.
  \item Przygotujesz grunt pod funkcje wyzszego rzedu i przekazywanie zachowania.
\end{learningobjectives}

\section{Mapa rozdzialu}
\begin{itemize}
  \item Powtorka: podpis funkcji, parametry domyslne, nazwane argumenty.
  \item Funkcje jako wyrazenia; single expression functions.
  \item Lambdy i typy funkcyjne.
  \item Funkcje wyzszego rzedu, inline, crossinline, noinline.
  \item Praca z kolekcjami i funkcjami standardowymi (`map`, `filter`, `fold`).
  \item Zadania i bledy z laboratorium (Lista2, Lista3).
\end{itemize}

\section{Material zrodlowy}
\begin{itemize}
  \item Slajdy: `PUM1/Wyk/Wyklad2.pdf` (funkcje) oraz `PUM1/Wyk/Wyklad4.pdf` dla rozwiniecia kolekcji.
  \item Notebooki: `PUM1/Listy/Lista2.ipynb`, `PUM1/Listy/Lista3.ipynb` (sekcje dotyczace lambd i funkcji kolekcyjnych).
  \item Dokumentacja Kotlin: sekcja "Functions" oraz "Higher-Order Functions".
\end{itemize}

\section{Struktura docelowa}
\subsection{Deklarowanie funkcji}
\begin{itemize}
  \item Sygnatura, typ zwracany, parametry z wartosciami domyslnymi.
  \item Argumenty nazwane i kolejnosc.
  \item Funkcje lokalne i zagniezdzone.
\end{itemize}

\subsection{Funkcje jako wyrazenia}
\begin{itemize}
  \item Single expression functions.
  \item Typ wywnioskowany vs jawny.
  \item Rola `Unit` i typow zwracanych.
\end{itemize}

\subsection{Lambdy i typy funkcyjne}
\begin{itemize}
  \item Skrotowa i pelna skladnia lambd.
  \item Parametr `it`, lambdy wieloliniowe.
  \item Referencje do funkcji (`::nazwa`).
\end{itemize}

\subsection{Funkcje wyzszego rzedu}
\begin{itemize}
  \item Funkcje przyjmujace i zwracajace funkcje.
  \item `inline`, `crossinline`, `noinline` - kiedy maja znaczenie.
  \item Praca z kolekcjami: `map`, `filter`, `flatMap`, `fold`, `reduce`.
\end{itemize}

\subsection{Case study z Listy}
\begin{itemize}
  \item Zadanie z `Lista2`: przetwarzanie danych tekstowych (TODO: wskazac konkretny zeszyt zadań).
  \item Zadanie z `Lista3`: transformacje kolekcji i walidacje.
  \item Najczestsze bledy: mutowanie kolekcji, `return` z lambdy, uzycie `forEach` vs `map`.
\end{itemize}

\section{Elementy do rozbudowy}
\begin{itemize}
  \item \textbf{Przyklad A}: aplikacja konsolowa podsumowujaca oceny (TODO: przygotowac kod i komentarz).
  \item \textbf{Przyklad B}: pipeline przetwarzania danych z API (TODO: bazowac na json w `Listy`).
  \item Zadania domowe: trzy krotkie problemy na map/filter/reduce (TODO: opisac).
  \item Pytania kontrolne: roznica `inline`/`noinline`, co zwraca `fold`, kiedy preferowac `sequence`.
\end{itemize}

% TODO: W kolejnym etapie uzupelnic tekst akapitami, wstawic kod z minted oraz dodac cytowania do dokumentacji.

