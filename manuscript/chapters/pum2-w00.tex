\chapter{Organizacja kursu i nawigacja Compose (Wykład 0)}

\section{Podstawowe informacje o kursie}
\subsection{Organizacja kursu}
Zajęcia obejmują łącznie 75 godzin pracy studenta, z czego 15 godzin przeznaczonych jest na wykłady (45 minutowy wykład na tydzień), a 30 godzin na zajęcia laboratoryjne. Dodatkowo przewidziano 30 godzin pracy własnej studenta, niezbędnej do samodzielnego opracowania materiału, przygotowania projektów oraz pogłębiania wiedzy. Kurs kończy się uzyskaniem 3 punktów ECTS. 

W trakcie zajęć wykorzystywane są materiały dydaktyczne dostępne pod adresem: \url{https://github.com/RafLew84/ProgUM}.  Obowiązkowa i zalecana literatura:
\begin{itemize}
\item Dokumentację języka Kotlin (\url{https://kotlinlang.org/docs/home.html}
) 
\item oficjalne kursy programowania aplikacji mobilnych dostępne na platformie Android Developers (\url{https://developer.android.com/courses}).
\end{itemize}

\subsection{Zasady zaliczenia}
\begin{itemize}
  \item Warunkiem zaliczenia laboratorium jest uzyskanie \textbf{oceny pozytywnej z list zadań}.
  \item Na zajęcia przewidzianych jest \textbf{5 list zadań}.
  \item Z każdej listy wystawiana jest \textbf{osobna ocena}.
  \item \textbf{Nie jest konieczne} zaliczenie wszystkich list aby otrzymać ocenę pozytywną z laboratorium. Dopuszczalne jest nieoddanie/niezaliczenie \textbf{jednej listy} - za tą listę otrzymuje się ocenę 2,0.
  \item Każda lista posiada informację o \textbf{liczbie} punktów wymaganych na konkretną ocenę
  \item Każda lista posiada \textbf{termin zwrotu}.
  \item Za \textbf{każdy tydzień opóźnienia} otrzymana ocena jest \textbf{obniżana o 1,0}.
  \item Listy oddawane są \textbf{podczas zajęć laboratoryjnych}. 
  \item Do każdej listy prowadzący \textbf{zadaje 4 pytania}.
  \item Liczba punktów za listę jest przyznawana na podstawie \textbf{poprawności wykonania zadań} oraz \textbf{odpowiedzi ustnej}.
  \item Ocena końcowa jest \textbf{średnią arytmetyczną} ze \textbf{wszystkich ocen} z list.
  \item Na ocenę 3,0 \textbf{wymagana jest} średnia co najmniej 3,0.
  \item Na zajęciach laboratoryjnych dopuszczalne są \textbf{trzy nieobecności nieusprawiedliwione}.
\end{itemize}

\subsection{Treści Programowe}
\begin{enumerate}
  \item Zasady zaliczenia, Treści Programowe, Zaawansowana Nawigacja.
  \item Wprowadzenie do Wielowątkowości: Coroutines. Wątek główny.
  \item Coroutines. Współbieżność, Równoległość, Asynchroniczność.
  \item Podstawy Architektury Aplikacji: Wzorce MVx (MVC, MVP, MVVM).
  \item Reaktywne Zarządzanie Stanem: Flow, StateFlow, SharedFlow.
  \item Zaawansowane Zarządzanie Stanem: withContext, StateIn, ShareIn, FlowOn, combine.
  \item Coroutines: Kanały - Asynchroniczna Wymiana Danych Między Coroutines.
  \item Zapis Danych do Pliku: SharedPreferences, DataStore.
  \item Baza Danych SQLite + ROOM: Entity, Dao, Database, CRUD, Operacje Asynchroniczne.
  \item Praca z Zewnętrznymi Źródłami Danych: Retrofit2, Operacje Asynchroniczne.
  \item Wstrzykiwanie Zależności: Dagger, Hilt.
  \item Czysta Architektura - Warstwa Domeny i Wzorzec Use Case.
  \item Wzorzec Single Source of Truth - Strategia Offline Caching.
  \item Backend w Chmurze: Wprowadzenie do Firebase i Firestore
\end{enumerate}

\subsection{Android Studio - Rozpoczęcie i konfiguracja projektu}
Podczas zajęć tworzymy projekty w Android Studio, w Kotlinie, z interfejsem budowanym w Jetpack Compose (bez widoków XML). Najprościej zacząć od kreatora: \texttt{New Project $\rightarrow$ Empty Activity}. Ustaw \texttt{Language: Kotlin}, \texttt{Minimum SDK: 28+} i zaakceptuj ustawienia. Taki szablon tworzy gotowy projekt z poprawną konfiguracją Compose i przykładowym \texttt{setContent \{ … \}}. 

\textbf{UWAGA!!!} Nie używaj Empty Views Activity - to rozpoczyna projekt oparty o \textit{widoki}, gdzie ui definiujemy jako pliki XML.

\subsection{Dodawanie zależności}
Wszystkie zależności dopisujemy w bloku \texttt{dependencies \{ … \}} pliku \texttt{build.gradle(.kts)(Module:App)}. Przykładowy plik:

\textbf{UWAGA!!!!} NIE KOPIUJ PLIKÓW KONFIGURACYJNYCH - zawierają informacje UNIKALNE DLA PROJEKTU !!!!

\begin{minted}{kotlin}
plugins {
    alias(libs.plugins.android.application)
    alias(libs.plugins.kotlin.android)
    alias(libs.plugins.kotlin.compose)
}

android {
    namespace = "com.example.test"
    compileSdk = 36

    defaultConfig {
        applicationId = "com.example.test"
        minSdk = 28
        targetSdk = 36
        versionCode = 1
        versionName = "1.0"

        testInstrumentationRunner = "androidx.test.runner.AndroidJUnitRunner"
    }

    buildTypes {
        release {
            isMinifyEnabled = false
            proguardFiles(
                getDefaultProguardFile("proguard-android-optimize.txt"),
                "proguard-rules.pro"
            )
        }
    }
    compileOptions {
        sourceCompatibility = JavaVersion.VERSION_11
        targetCompatibility = JavaVersion.VERSION_11
    }
    kotlinOptions {
        jvmTarget = "11"
    }
    buildFeatures {
        compose = true
    }
}

dependencies {
    // tutaj dodajemy zależności

    implementation(libs.androidx.core.ktx)
    implementation(libs.androidx.lifecycle.runtime.ktx)
    implementation(libs.androidx.activity.compose)
    implementation(platform(libs.androidx.compose.bom))
    implementation(libs.androidx.ui)
    implementation(libs.androidx.ui.graphics)
    implementation(libs.androidx.ui.tooling.preview)
    implementation(libs.androidx.material3)
    testImplementation(libs.junit)
    androidTestImplementation(libs.androidx.junit)
    androidTestImplementation(libs.androidx.espresso.core)
    androidTestImplementation(platform(libs.androidx.compose.bom))
    androidTestImplementation(libs.androidx.ui.test.junit4)
    debugImplementation(libs.androidx.ui.tooling)
    debugImplementation(libs.androidx.ui.test.manifest)
}
\end{minted}

\textbf{UWAGA!!!} Pamiętaj aby wykonać \textbf{synchronizację} (\ref{fig:figura-1}) projektu po jakiejkolwiek zmianie w plikach z katalogu \texttt{gradle}

\begin{figure}[htbp]
    \includegraphics{fig/pum2-w00-fig1.png}
    \caption{Okno programu z widokiem na plik konfiguracyjny.}
    \label{fig:figura-1}
\end{figure}

\subsection{Szuflada Nawigacyjna}
W tej części skupimy się na praktycznej implementacji nawigacji w aplikacji Jetpack Compose. Zakładając, znajomość podstawowych bloków konstrukcyjnych: NavHost i NavController, przyjrzymy się, jak zintegrować je z zaawansowanymi komponentami Material 3, takimi jak szuflada nawigacyjna.

Przeanalizujmy kod, który implementuje jeden z najczęstszych wzorców nawigacyjnych w aplikacjach mobilnych: szufladę nawigacyjną (znaną również jako \textit{hamburger menu})(\ref{fig:Figura 2}).

\begin{figure}[htbp]
    \includegraphics[width=0.7\textwidth]{fig/pum2-w00-fig2.png}
    \caption{Szuflada nawigacyjna.}
    \label{fig:Figura 2}
\end{figure}

Załączony kod to kompletna, choć minimalistyczna, aplikacja demonstrująca łączenie komponentów \texttt{Jetpack Navigation}, \texttt{Scaffold} oraz \texttt{ModalNavigationDrawer}. Przyjrzyjmy się jej kluczowym elementom krok po kroku.

Aby móc pracować z \texttt{compose navigation} musimy dodać odpowiednią \textbf{zależność} do projektu

\begin{minted}{kotlin}
dependencies {
    implementation("androidx.navigation:navigation-compose:2.9.2")
}
\end{minted}

Pierwszym elementem jest \textbf{centralizacja tras nawigacyjnych} w obiekcie \texttt{data object AppDestinations}.

\begin{minted}{kotlin}
data object AppDestinations {
    const val HOME = "home"
    const val PROFILE = "profile"
    const val SETTINGS = "settings"
}
\end{minted}


Centralizacja zapewnia bezpieczeństwo typów, ułatwia refaktoryzację (zmieniamy nazwę w jednym miejscu).

W aplikajci zagnieżdżamy kilku komponentów . Hierarchia wygląda następująco:

\begin{minted}{kotlin}
@OptIn(ExperimentalMaterial3Api::class)
@Composable
fun MainApp() {
    val navController = rememberNavController()
    val drawerState = rememberDrawerState(initialValue = DrawerValue.Closed)
    val scope = rememberCoroutineScope()

    ModalNavigationDrawer(
        drawerState = drawerState,
        drawerContent = {
            DrawerContent(navController = navController, drawerState = drawerState)
        }
    ) {
        Scaffold(
            topBar = {
                TopAppBar(
                    title = { Text("Aplikacja z Szufladą") },
                    navigationIcon = {
                        IconButton(onClick = {
                            scope.launch { drawerState.apply { if (isClosed) open() else close() } }
                        }) { Icon(Icons.Filled.Menu, contentDescription = "Menu") }
                    }
                )
            }
        ) { paddingValues ->
            NavHost(
                navController = navController,
                startDestination = AppDestinations.HOME,
                modifier = Modifier.padding(paddingValues)
            ) {
                composable(AppDestinations.HOME) { HomeScreen() }
                composable(AppDestinations.PROFILE) { ProfileScreen() }
                composable(AppDestinations.SETTINGS) { SettingsScreen() }
            }
        }
    }
}
\end{minted}

\begin{enumerate}
  \item \texttt{ModalNavigationDrawer}: Jest to komponent najwyższego poziomu, który zarządza logiką pokazywania i ukrywania wysuwanej szuflady. Przyjmuje on dwa kluczowe parametry:
  \begin{itemize}
    \item \texttt{drawerState}: Stan szuflady (otwarta/zamknięta).
    \item \texttt{drawerContent}: Funkcja kompozycyjna definiująca zawartość samej szuflady (\texttt{DrawerContent}).
  \end{itemize}
  \item \texttt{Scaffold}: Umieszczony wewnątrz \texttt{ModalNavigationDrawer}, \texttt{Scaffold} zapewnia standardową strukturę ekranu. W naszym przypadku używamy go do zdefiniowania \texttt{topBar} (górnego paska aplikacji).

  \item \texttt{NavHost}: Na końcu, wewnątrz \texttt{Scaffold}, umieszczamy nasz \texttt{NavHost}. To on zarządza faktyczną podmianą treści ekranu (\texttt{HomeScreen, ProfileScreen} itd.). Zwróćmy uwagę na kluczowe powiązanie: \texttt{modifier = Modifier.padding(paddingValues)}. Przekazujemy tu wypełnienie (\texttt{paddingValues}) otrzymane ze \texttt{Scaffold}, co zapewnia, że nasza treść nie zostanie przysłonięta przez \texttt{TopAppBar}.
\end{enumerate}

W funkcji \texttt{MainApp} zauważymy trzy kluczowe \textbf{zmienne stanu}.

\begin{minted}{kotlin}
val navController = rememberNavController()
val drawerState = rememberDrawerState(initialValue = DrawerValue.Closed)
val scope = rememberCoroutineScope()
\end{minted}

Pierwsze dwie są oczywiste: jedna zarządza \textbf{stosem nawigacji}, druga \textbf{stanem szuflady}. Ale dlaczego potrzebujemy \texttt{scope}?

Odpowiedź leży w naturze \texttt{drawerState}. Metody \texttt{drawerState.open()} i \texttt{drawerState.close()} są funkcjami zawieszającymi (\texttt{suspend functions}) - więcej o tych funkcjach w kolejnych rozdziałach. Nie można ich po prostu wywołać z dowolnego miejsca – muszą być uruchomione wewnątrz \textbf{korutyny}. Zmiana \texttt{drawerState} automatycznie spowoduje \textbf{rekompozycję} i wizualne otwarcie szuflady.

\begin{minted}{kotlin}
{scope.launch { drawerState.apply { if (isClosed) open() else close() } }})
\end{minted}

Przejdźmy do zarządzania nawigację z wnętrza szuflady. W funkcji \texttt{MainApp} dodaliśmy \texttt{ModalNavigationDrawer}, który posiada parametr \texttt{drawerContent} - jako ten parametr przekazujemy funkcję \texttt{DrawerContent}, zajrzyjmmy do jej wnętrza:

\begin{minted}{kotlin}
@OptIn(ExperimentalMaterial3Api::class)
@Composable
fun DrawerContent(navController: NavController, drawerState: DrawerState) {
    val scope = rememberCoroutineScope()

    ModalDrawerSheet {
        Column(modifier = Modifier.padding(16.dp)) {
            Text("Menu", style = MaterialTheme.typography.headlineSmall)
            Spacer(modifier = Modifier.height(16.dp))

            NavigationDrawerItem(
                icon = { Icon(Icons.Default.Home, contentDescription = "Strona główna") },
                label = { Text("Strona główna") },
                selected = false,
                onClick = {
                    navController.navigate(AppDestinations.HOME)
                    scope.launch { drawerState.close() }
                }
            )
            NavigationDrawerItem(
                icon = { Icon(Icons.Default.Person, contentDescription = "Profil") },
                label = { Text("Profil") },
                selected = false,
                onClick = {
                    navController.navigate(AppDestinations.PROFILE)
                    scope.launch { drawerState.close() }
                }
            )
            NavigationDrawerItem(
                icon = { Icon(Icons.Default.Settings, contentDescription = "Ustawienia") },
                label = { Text("Ustawienia") },
                selected = false,
                onClick = {
                    navController.navigate(AppDestinations.SETTINGS)
                    scope.launch { drawerState.close() }
                }
            )
        }
    }
}
\end{minted}

Wewnątrz \texttt{ModalDrawerSheet} definiujemy elementy będące wyświetlane w szufladzie jako \texttt{NavigationDrawerItem}. Jest to element zaprojektowany według tego samego wzorca (\textit{slot-based-layouts}). Posiada zdefiniowane \textit{sloty} w które można wstawić elementy (\texttt{icon, label, divider}), upraszczając tworzenie całego layoutu.
Po kliknięciu elementu w szufladzie wykonujemy dwie akcje:
\begin{itemize}
  \item \textbf{Nawigujemy}: Wywołujemy \texttt{navController.navigate()}, aby zmienić zawartość \texttt{NavHost}.
  \item \textbf{Zamykamy szufladę}: Uruchamiamy korutynę (\texttt{scope.launch}), aby wywołać \texttt{drawerState.close()}.
\end{itemize}

Jak widzimy, nawigacja w Jetpack Compose to znacznie więcej niż tylko wywoływanie \texttt{navController.navigate()}. To przemyślana integracja kontrolera nawigacji z innymi komponentami interfejsu, takimi jak \texttt{ModalNavigationDrawer}. Istotnym elementem jest zrozumienie, jak zarządzać wieloma stanami (\texttt{navController, drawerState}) oraz jak obsługiwać \textbf{asynchroniczne wywołania} (funkcje \texttt{suspend} do otwierania/zamykania szuflady) za pomocą odpowiednich narzędzi, takich jak \texttt{rememberCoroutineScope}.

Na kolejnych wykładach zagłębimy się w korutyny, które są fundamentem działania tego przykładu.


\subsubsection{pełny kod przykładu}

\begin{minted}{kotlin}
class MainActivity : ComponentActivity() {
    override fun onCreate(savedInstanceState: Bundle?) {
        super.onCreate(savedInstanceState)
        enableEdgeToEdge()
        setContent {
            MyApplicationTheme {
                MainApp()
            }
        }
    }
}

data object AppDestinations {
    const val HOME = "home"
    const val PROFILE = "profile"
    const val SETTINGS = "settings"
}

@Composable
fun HomeScreen() {
    Box(
        modifier = Modifier.fillMaxSize(),
        contentAlignment = Alignment.Center
    ) {
        Text("Strona Główna", fontSize = 24.sp)
    }
}

@Composable
fun ProfileScreen() {
    Box(
        modifier = Modifier.fillMaxSize(),
        contentAlignment = Alignment.Center
    ) {
        Text("Profil Użytkownika", fontSize = 24.sp)
    }
}

@Composable
fun SettingsScreen() {
    Box(
        modifier = Modifier.fillMaxSize(),
        contentAlignment = Alignment.Center
    ) {
        Text("Ustawienia", fontSize = 24.sp)
    }
}

@OptIn(ExperimentalMaterial3Api::class)
@Composable
fun MainApp() {
    val navController = rememberNavController()
    val drawerState = rememberDrawerState(initialValue = DrawerValue.Closed)
    val scope = rememberCoroutineScope()

    ModalNavigationDrawer(
        drawerState = drawerState,
        drawerContent = {
            DrawerContent(navController = navController, drawerState = drawerState)
        }
    ) {
        Scaffold(
            topBar = {
                TopAppBar(
                    title = { Text("Aplikacja z Szufladą") },
                    navigationIcon = {
                        IconButton(onClick = {
                            scope.launch { drawerState.apply { if (isClosed) open() else close() } }
                        }) { Icon(Icons.Filled.Menu, contentDescription = "Menu") }
                    }
                )
            }
        ) { paddingValues ->
            NavHost(
                navController = navController,
                startDestination = AppDestinations.HOME,
                modifier = Modifier.padding(paddingValues)
            ) {
                composable(AppDestinations.HOME) { HomeScreen() }
                composable(AppDestinations.PROFILE) { ProfileScreen() }
                composable(AppDestinations.SETTINGS) { SettingsScreen() }
            }
        }
    }
}



@OptIn(ExperimentalMaterial3Api::class)
@Composable
fun DrawerContent(navController: NavController, drawerState: DrawerState) {
    val scope = rememberCoroutineScope()

    ModalDrawerSheet {
        Column(modifier = Modifier.padding(16.dp)) {
            Text("Menu", style = MaterialTheme.typography.headlineSmall)
            Spacer(modifier = Modifier.height(16.dp))

            NavigationDrawerItem(
                icon = { Icon(Icons.Default.Home, contentDescription = "Strona główna") },
                label = { Text("Strona główna") },
                selected = false,
                onClick = {
                    navController.navigate(AppDestinations.HOME)
                    scope.launch { drawerState.close() }
                }
            )
            NavigationDrawerItem(
                icon = { Icon(Icons.Default.Person, contentDescription = "Profil") },
                label = { Text("Profil") },
                selected = false,
                onClick = {
                    navController.navigate(AppDestinations.PROFILE)
                    scope.launch { drawerState.close() }
                }
            )
            NavigationDrawerItem(
                icon = { Icon(Icons.Default.Settings, contentDescription = "Ustawienia") },
                label = { Text("Ustawienia") },
                selected = false,
                onClick = {
                    navController.navigate(AppDestinations.SETTINGS)
                    scope.launch { drawerState.close() }
                }
            )
        }
    }
}
\end{minted}

\subsection{Zagnieżdźona Nawigacja}

Przejdźmy do drugiego przykładu w którym zapoznamy się z ideą tworzenia zagnieżdżonych grafów nawigacyjnych.

Załóżmy, że nasza aplikacja ma więcej niż trzy proste ekrany. Prawie każda komercyjna aplikacja ma co najmniej dwa odrębne \textit{przepływy} (flows):
\begin{itemize}
  \item Przepływ Uwierzytelniania: Logowanie, Rejestracja, Resetowanie Hasła.
  \item Główny Przepływ Aplikacji: Ekran główny, Profil, Ustawienia, itd.
\end{itemize}

Problem polega na tym, że te dwa przepływy mają zupełnie inne zasady. Co najważniejsze: gdy użytkownik pomyślnie się zaloguje, powinien przejść do ekranu głównego, a cały przepływ uwierzytelniania powinien zniknąć z historii. Naciśnięcie przycisku \textit{Wstecz} na ekranie głównym nie powinno cofać do ekranu logowania, lecz zamykać aplikację. Osiągnąć to można za pomocą zagnieżdżonej nawigacji.

Przeanalizujmy prosty przykład pokazujący takie rozwiązanie. Zamiast traktować \texttt{NavHost} jak jeden wielki kontener, traktujemy go jak folder, który może zawierać zarówno pojedyncze pliki (ekrany), jak i inne foldery (zagnieżdżone grafy).

Pierwszym krokiem jest zdefiniowanie tras w obiekcie \texttt{AppDestinations}:

\begin{minted}{kotlin}
data object AppDestinations {
    // Grafy
    const val AUTH_GRAPH = "auth_graph"
    const val MAIN_APP_GRAPH = "main_app_graph"

    // Ekrany uwierzytelniania
    const val LOGIN = "login"
    const val REGISTER = "register"
    const val FORGOT_PASSWORD = "forgot_password"

    // Główne ekrany aplikacji
    const val WELCOME = "welcome"
    const val PROFILE = "profile"
}
\end{minted}

Zauważmy, że nasza aplikacja będzie miała dwa główne podgrafy: \texttt{AUTH\_GRAPH} i \texttt{MAIN\_APP\_GRAPH}

Spójrzmy na główny \texttt{NavHost} w \texttt{SimpleNestedNavApp}. Jest on uderzająco prosty:

\begin{minted}{kotlin}
NavHost(
    navController = navController,
    startDestination = AppDestinations.AUTH_GRAPH
) {
    // Graf uwierzytelniania (logowanie, rejestracja, itp.)
    authGraph(navController)

    // Główny graf aplikacji po zalogowaniu
    mainAppGraph(navController)
}
\end{minted}

Zauważmy:
\begin{enumerate}
\item \texttt{startDestination} nie jest ekranem. Jest to \texttt{AUTH\_GRAPH}, czyli cały zagnieżdżony graf. Aplikacja uruchamia się, wchodząc do \textit{folderu} uwierzytelniania.
\item Wewnątrz \texttt{NavHost} nie ma ani jednego composable() definiującego ekran. Zamiast tego, są tylko dwie funkcje (authGraph i mainAppGraph), które definiują całe grupy ekranów.
\end{enumerate}

Główny \texttt{NavHost} nie musi \textit{wiedzieć} nic o ekranie logowania czy profilu; musi tylko \textit{wiedzieć} o istnieniu \textit{przepływu uwierzytelniania} i \textit{przepływu głównego}.

W jaki sposób definiować grafy zagnieżdżone? Służy do tego funkcja kompozycyjna \texttt{navigation()}. Standardowo tworzy się funkcje rozszerzające dla \texttt{NavGraphBuilder}:

\begin{minted}{kotlin}
fun NavGraphBuilder.authGraph(navController: NavController) {
    navigation(
        startDestination = AppDestinations.LOGIN,
        route = AppDestinations.AUTH_GRAPH
    ) {
        composable(AppDestinations.LOGIN) {
            LoginScreen(navController) }
        composable(AppDestinations.REGISTER) {
            RegisterScreen(navController) }
        composable(AppDestinations.FORGOT_PASSWORD) {
            ForgotPasswordScreen(navController) }
    }
}

fun NavGraphBuilder.mainAppGraph(navController: NavController) {
    navigation(
        startDestination = AppDestinations.WELCOME,
        route = AppDestinations.MAIN_APP_GRAPH
    ) {
        composable(AppDestinations.WELCOME) {
            WelcomeScreen(navController) }
        composable(AppDestinations.PROFILE) {
            ProfileScreen(navController) }
    }
}
\end{minted}

Analiza tego bloku jest kluczowa:
\begin{itemize}
\item \texttt{fun NavGraphBuilder.authGraph(...)}: To czysta organizacja kodu. Zamiast zaśmiecać \texttt{NavHost}, grupujemy logikę w oddzielnej funkcji.
\item \texttt{navigation(...)}: To jest właściwy konstruktor zagnieżdżonego grafu.
\item \texttt{route = AppDestinations.AUTH\_GRAPH}: Nadajemy całemu \textit{folderowi} nazwę. Teraz możemy nawigować \textit{do} niego.
\item \texttt{startDestination = AppDestinations.LOGIN}: Definiujemy, który ekran jest domyślny wewnątrz tego grafu.
\end{itemize}

Identyczną strukturę ma \texttt{mainAppGraph}, który grupuje ekrany \texttt{WELCOME} i \texttt{PROFILE}.

Mamy dwa oddzielne światy: \texttt{AUTH\_GRAPH} i \texttt{MAIN\_APP\_GRAPH}. Jak przeskoczyć z jednego do drugiego i – co najważniejsze – posprzątać po sobie?

Spójrzmy na funkcję \texttt{navigateToMainApp()}, wywoływaną po pomyślnym logowaniu:

\begin{minted}{kotlin}
fun NavController.navigateToMainApp() {
    this.navigate(AppDestinations.MAIN_APP_GRAPH) {
        popUpTo(AppDestinations.AUTH_GRAPH) {
            inclusive = true
        }
    }
}
\end{minted}

To jest najważniejszy fragment kodu w całym przykładzie. Rozbijmy go na części:
\begin{itemize}
\item \texttt{navigate(AppDestinations.MAIN\_APP\_GRAPH)}: Mówimy \textit{nawiguj do grafu głównego}. \texttt{NavController} automatycznie skieruje nas do \texttt{startDestination} tego grafu (czyli \texttt{WELCOME}).
\item \texttt{popUpTo(AppDestinations.AUTH\_GRAPH)}: To jest polecenie \textit{sprzątające}. Powraca na stosie aż znajdzie \texttt{AUTH\_GRAPH}
\item \texttt{inclusive = true}: Po znalezieniu na stosie \texttt{AUTH\_GRAPH} jest on również usuwany.
\end{itemize}

Co się dzieje w praktyce? Użytkownik klika \textit{Zaloguj}. \texttt{NavController}:
\begin{itemize}
\item Znajduje na stosie powrotu graf \texttt{AUTH\_GRAPH}.
\item Usuwa ze stosu \texttt{LOGIN, REGISTER, FORGOT\_PASSWORD...} i sam \texttt{AUTH\_GRAPH}.
\item Dodaje na stos \texttt{MAIN\_APP\_GRAPH} (z ekranem \texttt{WELCOME}).
\end{itemize}

Stos powrotu jest czysty. Zawiera tylko \texttt{MAIN\_APP\_GRAPH}. Jeśli użytkownik naciśnie teraz przycisk \textit{Wstecz}, nie wróci do ekranu logowania. Opuści aplikację. Osiągnęliśmy dokładnie taki przepływ, jakiego oczekują użytkownicy.

Zagnieżdżona nawigacja to nie jest skomplikowana funkcja dla zaawansowanych. To podstawowe narzędzie do organizacji w każdej aplikacji, która ma więcej niż jeden logiczny przepływ. Jak widzieliśmy w kodzie, ten wzorzec zapewnia trzy kluczowe korzyści:
\begin{enumerate}
\item Organizację: Grupuje powiązane ekrany.
\item Modularność: Utrzymuje główny \texttt{NavHost} czysty i pozwala definiować przepływy w oddzielnych funkcjach.
\item Kontrolę nad Stosem Powrotu: Umożliwia nawigowanie między całymi przepływami i usuwanie ich z historii jednym poleceniem.
\end{enumerate}

\subsubsection{Pełny kod przykładu}

\begin{minted}{kotlin}
class MainActivity : ComponentActivity() {
    override fun onCreate(savedInstanceState: Bundle?) {
        super.onCreate(savedInstanceState)
        enableEdgeToEdge()
        setContent {
            NestedComposeGraphTheme {
                SimpleNestedNavApp()
            }
        }
    }
}

data object AppDestinations {
    // Grafy
    const val AUTH_GRAPH = "auth_graph"
    const val MAIN_APP_GRAPH = "main_app_graph"

    // Ekrany autentykacji
    const val LOGIN = "login"
    const val REGISTER = "register"
    const val FORGOT_PASSWORD = "forgot_password"

    // Główne ekrany aplikacji
    const val WELCOME = "welcome"
    const val PROFILE = "profile"
}

@Composable
fun SimpleNestedNavApp() {
    val navController = rememberNavController()
    NavHost(
        navController = navController,
        startDestination = AppDestinations.AUTH_GRAPH
    ) {
        // Graf autentykacji (logowanie, rejestracja, itp.)
        authGraph(navController)

        // Główny graf aplikacji po zalogowaniu
        mainAppGraph(navController)
    }
}

fun NavGraphBuilder.authGraph(navController: NavController) {
    navigation(
        startDestination = AppDestinations.LOGIN,
        route = AppDestinations.AUTH_GRAPH
    ) {
        composable(AppDestinations.LOGIN) {
            LoginScreen(navController) }
        composable(AppDestinations.REGISTER) {
            RegisterScreen(navController) }
        composable(AppDestinations.FORGOT_PASSWORD) {
            ForgotPasswordScreen(navController) }
    }
}

fun NavGraphBuilder.mainAppGraph(navController: NavController) {
    navigation(
        startDestination = AppDestinations.WELCOME,
        route = AppDestinations.MAIN_APP_GRAPH
    ) {
        composable(AppDestinations.WELCOME) {
            WelcomeScreen(navController) }
        composable(AppDestinations.PROFILE) {
            ProfileScreen(navController) }
    }
}

@Composable
fun LoginScreen(navController: NavController) {
    Column(
        modifier = Modifier.fillMaxSize().padding(16.dp),
        verticalArrangement = Arrangement.Center,
        horizontalAlignment = Alignment.CenterHorizontally
    ) {
        Text("Logowanie", fontSize = 28.sp)
        Spacer(Modifier.height(24.dp))
        Button(onClick = { navController.navigateToMainApp() }) {
            Text("Zaloguj")
        }
        Spacer(Modifier.height(12.dp))
        Button(onClick = { navController.navigate(AppDestinations.REGISTER) }) {
            Text("Przejdź do Rejestracji")
        }
        Spacer(Modifier.height(12.dp))
        TextButton(onClick = { navController.navigate(AppDestinations.FORGOT_PASSWORD) }) {
            Text("Zapomniałem hasła")
        }
    }
}

@Composable
fun RegisterScreen(navController: NavController) {
    Column(
        modifier = Modifier.fillMaxSize().padding(16.dp),
        verticalArrangement = Arrangement.Center,
        horizontalAlignment = Alignment.CenterHorizontally
    ) {
        Text("Rejestracja", fontSize = 28.sp)
        Spacer(Modifier.height(24.dp))
        Button(onClick = { navController.navigateToMainApp() }) {
            Text("Zarejestruj i zaloguj")
        }
        Spacer(Modifier.height(12.dp))
        TextButton(onClick = { navController.popBackStack() }) {
            Text("Wróć do logowania")
        }
    }
}

@Composable
fun ForgotPasswordScreen(navController: NavController) {
    Column(
        modifier = Modifier.fillMaxSize().padding(16.dp),
        verticalArrangement = Arrangement.Center,
        horizontalAlignment = Alignment.CenterHorizontally
    ) {
        Text("Resetowanie Hasła", fontSize = 24.sp, textAlign = TextAlign.Center)
        Spacer(Modifier.height(24.dp))
        Button(onClick = { navController.popBackStack() }) {
            Text("Powrót")
        }
    }
}
@Composable
fun WelcomeScreen(navController: NavController) {
    Column(
        modifier = Modifier.fillMaxSize().padding(16.dp),
        verticalArrangement = Arrangement.Center,
        horizontalAlignment = Alignment.CenterHorizontally
    ) {
        Text("Witaj w Aplikacji! 🎉", fontSize = 28.sp)
        Spacer(Modifier.height(24.dp))
        Button(onClick = { navController.navigate(AppDestinations.PROFILE) }) {
            Text("Zobacz mój profil")
        }
    }
}

@Composable
fun ProfileScreen(navController: NavController) {
    Column(
        modifier = Modifier.fillMaxSize().padding(16.dp),
        verticalArrangement = Arrangement.Center,
        horizontalAlignment = Alignment.CenterHorizontally
    ) {
        Text("Ekran Profilu 🧑‍💻", fontSize = 28.sp)
        Spacer(Modifier.height(24.dp))
        Button(onClick = { navController.popBackStack() }) {
            Text("Wróć do ekranu powitalnego")
        }
    }
}

fun NavController.navigateToMainApp() {
    this.navigate(AppDestinations.MAIN_APP_GRAPH) {
        popUpTo(AppDestinations.AUTH_GRAPH) {
            inclusive = true
        }
    }
}
\end{minted}
