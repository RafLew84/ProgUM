\chapter{Wyklad 0: Organizacja kursu i nawigacja Compose}

\begin{learningobjectives}
  \item Zrozumiesz zasady zaliczenia PUM2 oraz wymagane narzedzia.
  \item Przygotujesz srodowisko Android Studio na potrzeby zaawansowanych zajec (Compose + Hilt + Firebase).
  \item Zaplanujesz aplikacje przykladowa wykorzystywana w dalszych rozdzialach, ze szczegolnym naciskiem na Compose Navigation.
\end{learningobjectives}

\section{Mapa rozdzialu}
\begin{itemize}
  \item Organizacja PUM2: cele, harmonogram, forma zaliczenia, roznice wzgledem PUM1.
  \item Przeglad narzedzi: Android Studio (aktualna wersja), emulator, Firebase CLI, Postman/Insomnia.
  \item Konfiguracja projektu startowego: Compose Material 3, Hilt, modul app.
  \item Compose Navigation: podstawowe pojecia (NavHost, NavController, graf, destynacje, argumenty), integracja z scaffold.
  \item Przeglad notebookow `Listy` dedykowanych nawigacji (\texttt{W1-Drawer.ipynb}, \texttt{W1-Nested\_graph.ipynb}).
  \item Plan rozwoju aplikacji "Catalog App" w trakcie PUM2.
\end{itemize}

\section{Material zrodlowy}
\begin{itemize}
  \item Slajdy: `PUM2/Wyk/Wyklad1.pdf` (organizacja + Compose Navigation) oraz `PUM1/Wyk/Wyklad11.pdf` jako powtorka.
  \item Notebooki: \texttt{PUM2/Listy/W1-Drawer.ipynb}, \texttt{PUM2/Listy/W1-Nested\_graph.ipynb}.
  \item Dokumentacja: Compose Navigation (developer.android.com/jetpack/compose/navigation).
\end{itemize}

\section{Struktura docelowa}
\subsection{Organizacja kursu}
\begin{itemize}
  \item Cele PUM2: przejscie od podstaw Compose do pelnej aplikacji produkcyjnej.
  \item Zasady zaliczenia, wymagania projektowe i laboratoryjne.
  \item Repozytoria oraz sposoby dzielenia sie kodem (Git, GitHub Classroom, itp.).
\end{itemize}

\subsection{Srodowisko pracy}
\begin{itemize}
  \item Wersje narzedzi (Android Studio, gradle, Kotlin, Java, Firebase CLI).
  \item Konfiguracja emulatora i urzadzen fizycznych.
  \item Kontrola konfiguracji (gradle.properties, signing configs, sekretne klucze API).
\end{itemize}

\subsection{Projekt startowy}
\begin{itemize}
  \item Struktura modulow i pakietow.
  \item Setup Compose Material 3 + theming.
  \item Dodanie dependency Hilt, Navigation, Retrofit.
  \item Szkic ekranu startowego i bottom navigation (jesli wykorzystywana).
\end{itemize}

\subsection{Compose Navigation}
\begin{itemize}
  \item NavHost i konfiguracja grafu nawigacji.
  \item Destynacje, parametry, deep linki.
  \item Nested graphs, bottom navigation, drawer navigation.
  \item Integracja z ViewModel (state hoisting, savedStateHandle).
\end{itemize}

\subsection{Plan aplikacji przykladowej}
\begin{itemize}
  \item Zakres funkcjonalny (lista -> szczegoly -> ulubione -> ustawienia).
  \item Mapowanie przyszlych rozdzialow PUM2 na rozwijane funkcje.
  \item Przygotowanie backlogu i milestone'ow (np. offline caching, DI, Firestore).
\end{itemize}

\section{Elementy do rozbudowy}
\begin{itemize}
  \item \textbf{Przyklad A}: konfiguracja Compose Navigation w projekcie startowym (TODO: wkleic kod `NavHost`).
  \item \textbf{Przyklad B}: nested navigation (drawer + bottom nav) w oparciu o notebooki W1 (TODO: zrzuty i komentarz).
  \item Lista kontrolna setupu (TODO: tabela z krokami instalacji narzedzi).
  \item Pytania kontrolne: roznica NavGraph vs NavHost, jak przekazywac argumenty, jak testowac nawigacje.
\end{itemize}

% TODO: uzupelnic tresc narracyjna i przyklady kodu w kolejnych etapach.
