\chapter{Wyklad 13: Wprowadzenie do Firebase Firestore i pracy w chmurze}

\begin{learningobjectives}
  \item Skonfigurujesz Firebase w projekcie Android i przygotujesz build varianty.
  \item Poznasz podstawy modelowania danych w Firestore oraz zasady zapytan.
  \item Zaprojektujesz synchronizacje danych online/offline z uwzglednieniem bezpieczenstwa.
\end{learningobjectives}

\section{Mapa rozdzialu}
\begin{itemize}
  \item Firebase setup, konfiguracja projektu i narzedzi CLI.
  \item Firestore: kolekcje, dokumenty, zapytania, indeksy.
  \item Bezpieczenstwo: rules, auth, limity.
  \item Offline persistence i synchronizacja.
  \item Integracja z MVVM i Compose.
\end{itemize}

\section{Material zrodlowy}
\begin{itemize}
  \item firebase.google.com/docs/firestore (docelowe sekcje do zebrania cytatow).
  \item Przykładowe notebooki i repozytoria (TODO: wskazac konkretne zrodla).
  \item PUM2/Wyk/Wyklad13.pdf (uzupelnic material dedykowany Firestore).
\end{itemize}

\subsection{Konfiguracja Firebase}
\begin{itemize}
  \item Tworzenie projektu, plik google-services.json, gradle plugin.
  \item Debug vs release, emulator vs production.
  \item Integracja z modulami i secret management.
\end{itemize}

\subsection{Model danych Firestore}
\begin{itemize}
  \item Kolekcje, dokumenty, subkolekcje.
  \item Denormalizacja i ograniczenia zapytan.
  \item Zapytania, indeksery, paginacja.
\end{itemize}

\subsection{Bezpieczenstwo i synchronizacja}
\begin{itemize}
  \item Rules, custom claims i autoryzacja.
  \item Offline cache, conflict resolution, retries.
  \item Monitoring billingu i limitow.
\end{itemize}

\subsection{Case study}
\begin{itemize}
  \item Integracja katalogu produktow z Firestore.
  \item Zadanie: migracja danych lokalnych do chmury.
  \item Checklist: release readiness, testy e2e, backup.
\end{itemize}

\section{Elementy do rozbudowy}
\begin{itemize}
  \item Przyklad A: fetch i listen na kolekcji produktow (TODO).
  \item Przyklad B: rules dla kolekcji z uprawnieniami rolami (TODO).
  \item Zadania: zaprojektuj strukture kolekcji dla ulubionych (TODO).
  \item Pytania: limity Firestore, zasady offline cache, koszt zapytan (TODO).
\end{itemize}

% TODO: uzupelnic narracje, przyklady kodu i cytowania na etapie draftu.

