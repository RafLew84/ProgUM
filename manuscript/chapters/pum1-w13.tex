\chapter{Wyklad 13: Wzorce projektowe - behawioralne}

\begin{learningobjectives}
  \item Poznasz wzorce behawioralne i ich adaptacje w Kotlinie.
  \item Zastosujesz wzorce do organizacji logiki interakcji i przeplywu danych.
  \item Przeanalizujesz przypadki uzycia w projekcie katalogowym.
\end{learningobjectives}

\section{Mapa rozdzialu}
\begin{itemize}
  \item Observer, Strategy, Command, Chain of Responsibility.
  \item State, Memento, Mediator i ich adaptacje w Kotlinie.
  \item Implementacje z wykorzystaniem Flow i coroutines.
  \item Integracja wzorcow z Compose i MVVM.
  \item Antywzorce i pułapki implementacyjne.
\end{itemize}

\section{Material zrodlowy}
\begin{itemize}
  \item PUM1/Wyk/Wyklad13.pdf.
  \item \texttt{PUM2/Listy/W5-SharedFlow\_basic.ipynb}.
  \item refactoring.guru/design-patterns (sekcja behavioural).
\end{itemize}

\subsection{Przeglad wzorcow}
\begin{itemize}
  \item Observer w Flow i LiveData.
  \item Strategy dla polityk filtrowania danych.
  \item Command/UseCase jako enkapsulacja operacji.
\end{itemize}

\subsection{Stan i interakcje}
\begin{itemize}
  \item State pattern dla ekranow i dialogow.
  \item Memento dla operacji undo/redo.
  \item Mediator dla koordynacji komponentow UI.
\end{itemize}

\subsection{Implementacje w Kotlinie}
\begin{itemize}
  \item Uzycie sealed class i instrukcji when.
  \item Coroutines i Channels jako event bus.
  \item Testowanie wzorcow behawioralnych.
\end{itemize}

\subsection{Case study}
\begin{itemize}
  \item Projekt logiki filtrow w katalogu produktow.
  \item Zadanie: Chain of Responsibility dla walidacji.
  \item Checklist: rozpoznawanie antywzorcow w modulach UI.
\end{itemize}

\section{Elementy do rozbudowy}
\begin{itemize}
  \item Przyklad A: implementacja Strategy dla sortowania listy (TODO).
  \item Przyklad B: Chain of Responsibility dla walidacji formularza (TODO).
  \item Zadania: adaptacja wzorca Observer do Flow (TODO).
  \item Pytania: roznica Strategy i State, kiedy zastosowac Mediator (TODO).
\end{itemize}

% TODO: uzupelnic narracje, przyklady kodu i cytowania na etapie draftu.
