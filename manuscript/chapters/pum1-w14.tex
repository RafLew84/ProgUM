\chapter{Wyklad 14: Aplikacje hybrydowe, SharedPreferences i widzety}

\begin{learningobjectives}
  \item Zrozumiesz role aplikacji hybrydowych i WebView w ekosystemie Android.
  \item Porownasz SharedPreferences i DataStore oraz zaplanujesz migracje ustawien.
  \item Przygotujesz plan wdrozenia widzetu na pulpit z kontrola aktualizacji.
\end{learningobjectives}

\section{Mapa rozdzialu}
\begin{itemize}
  \item Aplikacje hybrydowe, WebView, bridge JavaScript.
  \item SharedPreferences i DataStore - podobienstwa i roznice.
  \item Widzet na pulpit: AppWidgetProvider, RemoteViews, aktualizacje.
  \item Bezpieczenstwo danych lokalnych i polityki sklepu.
  \item Zastosowania w projekcie kursowym (np. ulubione produkty).
\end{itemize}

\section{Material zrodlowy}
\begin{itemize}
  \item PUM1/Wyk/Wyklad14.pdf.
  \item \texttt{PUM2/Listy/W8-SharedPreferences\_basics.ipynb}, \texttt{PUM2/Listy/W8-PreferencesDataStore\_basics.ipynb}.
  \item developer.android.com/guide/topics/appwidgets.
\end{itemize}

\subsection{Aplikacje hybrydowe}
\begin{itemize}
  \item WebView konfiguracja i bezpieczenstwo.
  \item JavaScriptInterface i komunikacja z Kotlinem.
  \item Strategie offline dla aplikacji hybrydowych.
\end{itemize}

\subsection{SharedPreferences vs DataStore}
\begin{itemize}
  \item API SharedPreferences, commit vs apply.
  \item DataStore Preferences i Proto, migracje.
  \item Przechowywanie ustawien w Compose i ViewModel.
\end{itemize}

\subsection{Widgety na pulpit}
\begin{itemize}
  \item AppWidgetProvider, RemoteViews, konfiguracja layoutu.
  \item Aktualizacje periodiczne i event driven.
  \item Integracja z danymi i testowanie widgetow.
\end{itemize}

\subsection{Case study}
\begin{itemize}
  \item Widget ulubionych produktow.
  \item Zadanie: personalizacja ustawien UI i synchronizacja.
  \item Checklist: wydajnosc, bateria, zasady publikacji.
\end{itemize}

\section{Elementy do rozbudowy}
\begin{itemize}
  \item Przyklad A: implementacja widgetu listy ulubionych (TODO).
  \item Przyklad B: migracja SharedPreferences do DataStore (TODO).
  \item Zadania: dodanie trybu offline dla WebView (TODO).
  \item Pytania: kiedy wybrac DataStore Proto, jak zabezpieczyc JavaScriptInterface (TODO).
\end{itemize}

% TODO: uzupelnic narracje, przyklady kodu i cytowania na etapie draftu.
