\chapter{Wyklad 12: Offline caching i Single Source of Truth}

\begin{learningobjectives}
  \item Zrozumiesz wzorzec Single Source of Truth i jego zalety.
  \item Zaprojektujesz warstwe cache laczaca Room i Retrofit.
  \item Przygotujesz strategie synchronizacji danych offline/online.
\end{learningobjectives}

\section{Mapa rozdzialu}
\begin{itemize}
  \item Definicja SSOT i oczekiwane korzysci.
  \item Network Bound Resource pattern.
  \item Synchronizacja i polityki odswiezania.
  \item Mutex, concurrency, atomic updates.
  \item Error handling i fallback.
\end{itemize}

\section{Material zrodlowy}
\begin{itemize}
  \item PUM2/Wyk/Wyklad13.pdf.
  \item PUM2/Listy/W13-OfflineCachingBasics.ipynb, PUM2/Listy/W13-ActorDesignPatternOfflineCachingBasics.ipynb.
  \item developer.android.com/topic/performance/vitals/offline.
\end{itemize}

\subsection{Single Source of Truth}
\begin{itemize}
  \item Definicja i oczekiwane korzysci.
  \item Powiazanie Room i zrodel sieciowych.
  \item Schemat przeplywu danych w SSOT.
\end{itemize}

\subsection{Offline caching}
\begin{itemize}
  \item NetworkBoundResource wzorzec.
  \item Polityki odswiezania: stale while revalidate, timed refresh.
  \item Mutex, synchronizacja, atomic updates.
\end{itemize}

\subsection{Synchronizacja}
\begin{itemize}
  \item Strategie sync: push, pull, hybrid.
  \item Rozwiazywanie konfliktow danych.
  \item Monitoring, retry, metrics.
\end{itemize}

\subsection{Case study}
\begin{itemize}
  \item Cache dla katalogu produktow.
  \item Zadanie: implementacja strumienia SSOT.
  \item Checklist: fallback, logowanie, obserwowalnosc.
\end{itemize}

\section{Elementy do rozbudowy}
\begin{itemize}
  \item Przyklad A: NetworkBoundResource implementacja (TODO).
  \item Przyklad B: offline refresh flow z Mutex (TODO).
  \item Zadania: scenariusze sync do testow (TODO).
  \item Pytania: definicja SSOT, jak rozwiazywac konflikty (TODO).
\end{itemize}

% TODO: uzupelnic narracje, przyklady kodu i cytowania na etapie draftu.

