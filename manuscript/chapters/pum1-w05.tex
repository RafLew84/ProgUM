\chapter{Wyklad 5: Klasy, data/sealed classes i dziedziczenie}

\begin{learningobjectives}
  \item Utrwalisz zasady projektowania klas w Kotlinie.
  \item Zastosujesz data class, sealed class oraz enum class w modelu domenowym.
  \item Zaprojektujesz hierarchie odpornych na bledy struktur danych dla aplikacji mobilnej.
\end{learningobjectives}

\section{Mapa rozdzialu}
\begin{itemize}
  \item Konstrukcja klas, konstruktory, wlasciwosci i inicjalizacja.
  \item Data class i sealed class jako fundament modelu domenowego.
  \item Dziedziczenie, interfejsy, klasy abstrakcyjne.
  \item Null safety, kopiowanie obiektow i immutability.
  \item Modele danych w przeplywie API -> domena -> UI.
\end{itemize}

\section{Material zrodlowy}
\begin{itemize}
  \item PUM1/Wyk/Wyklad5.pdf.
  \item PUM1/Listy/Lista3.ipynb.
  \item kotlinlang.org/docs/classes.html.
\end{itemize}

\subsection{Konstrukcja klas}
\begin{itemize}
  \item Primary i secondary constructors, init blocks.
  \item Widocznosc pol i enkapsulacja.
  \item Composition over inheritance w praktyce.
\end{itemize}

\subsection{Typy specjalne}
\begin{itemize}
  \item Data class: componentN, copy, equals/hashCode.
  \item Sealed class vs enum class, modelowanie stanow i zdarzen.
  \item Inline/value class (wzmianka) i optymalizacja.
\end{itemize}

\subsection{Model domenowy}
\begin{itemize}
  \item Mapowanie DTO -> domain -> ui state.
  \item Null safety, sealed hierarchie dla error handling.
  \item Testowanie modeli i rozszerzen.
\end{itemize}

\subsection{Case study}
\begin{itemize}
  \item Projekt stanow ekranow Compose.
  \item Zadania z Listy: praca z data class i dziedziczeniem.
  \item Lista kontrolna atrybutow i walidacji danych.
\end{itemize}

\section{Elementy do rozbudowy}
\begin{itemize}
  \item Przyklad A: struktura modelu domenowego aplikacji katalogowej (TODO).
  \item Przyklad B: sealed interface do obslugi wynikow API (TODO).
  \item Zadania: refaktoryzacja hierarchii klas w zadaniu laboratoryjnym (TODO).
  \item Pytania: kiedy preferowac sealed class, jak dziala copy w data class (TODO).
\end{itemize}

% TODO: uzupelnic narracje, przyklady kodu i cytowania na etapie draftu.

