\chapter{Wyklad 7: Inicjalizacja i delegowanie wlasnosci}

\begin{learningobjectives}
  \item Opanujesz mechanizmy inicjalizacji w Kotlinie (lazy, lateinit, init).
  \item Wykorzystasz property delegation do kontrolowania stanu i walidacji.
  \item Przygotujesz wzorce inicjalizacji dla komponentow Android i Compose.
\end{learningobjectives}

\section{Mapa rozdzialu}
\begin{itemize}
  \item Kolejnosc inicjalizacji i bezpieczenstwo w Kotlinie.
  \item lateinit vs lazy, inicjalizacja na zadanie.
  \item Delegowane wlasciwosci: observable, vetoable, notNull.
  \item Delegacja do map i pamieci trwalej.
  \item Inicjalizacja ViewModel i zaleznosci w Androidzie.
\end{itemize}

\section{Material zrodlowy}
\begin{itemize}
  \item PUM1/Wyk/Wyklad7.pdf.
  \item PUM1/Listy/Lista4.ipynb.
  \item kotlinlang.org/docs/delegated-properties.html.
\end{itemize}

\subsection{Mechanizmy inicjalizacji}
\begin{itemize}
  \item Primary i secondary constructors, init blocks.
  \item lateinit, lazy i scenariusze uzycia.
  \item Bezpieczenstwo w wielowatkowosci i synchronizacja.
\end{itemize}

\subsection{Delegowane wlasciwosci}
\begin{itemize}
  \item observable, vetoable, notNull.
  \item Delegacja do map (np. argumenty fragmentow).
  \item Tworzenie wlasnych delegatow.
\end{itemize}

\subsection{Inicjalizacja w Android/Compose}
\begin{itemize}
  \item ViewModel i SavedStateHandle.
  \item Delegacja stanu w Compose (remember, rememberSaveable).
  \item Zarzadzanie cyklem zycia i cleanup.
\end{itemize}

\subsection{Case study}
\begin{itemize}
  \item Zadanie: konfiguracja ustawien aplikacji i lazy inicjalizacja zasobow.
  \item Analiza bledow race condition przy lazy.
  \item Checklist inicjalizacji komponentow.
\end{itemize}

\section{Elementy do rozbudowy}
\begin{itemize}
  \item Przyklad A: delegacja observable dla stanu UI (TODO).
  \item Przyklad B: wlasny delegat do przechowywania w SharedPreferences (TODO).
  \item Zadania: refaktoryzacja inicjalizacji w zadaniu laboratoryjnym (TODO).
  \item Pytania: kiedy uzywac lateinit, jak zapewnic thread safety (TODO).
\end{itemize}

% TODO: uzupelnic narracje, przyklady kodu i cytowania na etapie draftu.

