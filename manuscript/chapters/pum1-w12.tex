\chapter{Wyklad 12: Wzorce projektowe - strukturalne i kreacyjne}

\begin{learningobjectives}
  \item Przypomnisz najwazniejsze wzorce strukturalne i kreacyjne.
  \item Zmapujesz wzorce na idiomy Kotlin i komponenty Compose.
  \item Przygotujesz rekomendacje zastosowan w projekcie kursowym.
\end{learningobjectives}

\section{Mapa rozdzialu}
\begin{itemize}
  \item Builder, Factory, Singleton, Prototype.
  \item Adapter, Facade, Composite, Decorator.
  \item Zastosowanie wzorcow w Kotlinie i Compose.
  \item Antywzorce i naduzycia wzorcow.
  \item Przyklady z zadan laboratoryjnych i projektu.
\end{itemize}

\section{Material zrodlowy}
\begin{itemize}
  \item PUM1/Wyk/Wyklad12.pdf.
  \item PUM1/Listy/Lista6.ipynb.
  \item refactoring.guru/design-patterns (sekcje open source).
\end{itemize}

\subsection{Wzorce kreacyjne}
\begin{itemize}
  \item Builder, Factory Method, Abstract Factory.
  \item Zastosowanie w DI oraz przy tworzeniu ViewModel.
  \item Kotlin features: DSL, named parameters, default args.
\end{itemize}

\subsection{Wzorce strukturalne}
\begin{itemize}
  \item Adapter dla integracji API, Facade dla modulow.
  \item Decorator w Compose (modifier pattern).
  \item Composite dla hierarchii UI i danych.
\end{itemize}

\subsection{Dobre praktyki}
\begin{itemize}
  \item Kiedy stosowac wzorce vs prostsze rozwiazania.
  \item Testowalnosc i utrzymanie kodu wzorcowego.
  \item Integracja z clean architecture i DI.
\end{itemize}

\subsection{Case study}
\begin{itemize}
  \item Analiza projektu katalogu: gdzie uzyc Facade.
  \item Zadanie: implementacja buildera dla konfigurowalnego UI.
  \item Checklist: rozpoznawanie wzorcow i antywzorcow w kodzie.
\end{itemize}

\section{Elementy do rozbudowy}
\begin{itemize}
  \item Przyklad A: builder DSL dla konfiguracji ekranu (TODO).
  \item Przyklad B: adapter dla zewnetrznego API (TODO).
  \item Zadania: zidentyfikuj wzorce w przykladowym module (TODO).
  \item Pytania: kiedy uzyc decorator w Compose, roznice builder vs factory (TODO).
\end{itemize}

% TODO: uzupelnic narracje, przyklady kodu i cytowania na etapie draftu.

