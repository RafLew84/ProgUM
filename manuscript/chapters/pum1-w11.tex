\chapter{Wyklad 11: Nawigacja w aplikacji i Compose Navigation}

\begin{learningobjectives}
  \item Zbudujesz graf nawigacji w Compose z wykorzystaniem NavHost i NavController.
  \item Zarzadzisz parametrami destynacji, deep linkami oraz nested graphs.
  \item Zaprojektujesz stack nawigacyjny dla aplikacji katalogowej. 
\end{learningobjectives}

\section{Mapa rozdzialu}
\begin{itemize}
  \item Podstawy nawigacji: NavController, NavHost, NavGraph.
  \item Destynacje, argumenty i serializacja danych.
  \item Nested graphs, bottom navigation, drawer navigation.
  \item Integracja z ViewModel, SavedStateHandle i back stackiem.
  \item Testowanie nawigacji w Compose.
\end{itemize}

\section{Material zrodlowy}
\begin{itemize}
  \item PUM1/Wyk/Wyklad11.pdf.
  \item PUM2/Listy/W1-Drawer.ipynb, PUM2/Listy/W1-Nested\_graph.ipynb.
  \item developer.android.com/jetpack/compose/navigation.
\end{itemize}

\subsection{Podstawy nawigacji}
\begin{itemize}
  \item Konfiguracja NavHost i start destination.
  \item Dodawanie destynacji i argumentow.
  \item Back stack i zarzadzanie powrotami.
\end{itemize}

\subsection{Zaawansowana nawigacja}
\begin{itemize}
  \item Nested graphs i dynamiczne grafy.
  \item Bottom navigation, drawer, top-level destinations.
  \item Deep linki i integracja z systemem.
\end{itemize}

\subsection{Nawigacja a architektura}
\begin{itemize}
  \item Wspolpraca z ViewModel i state holderami.
  \item SavedStateHandle i przekazywanie parametrow.
  \item Testowanie nawigacji z Compose UI test.
\end{itemize}

\subsection{Case study}
\begin{itemize}
  \item Mapa nawigacji katalogu produktow.
  \item Zadanie: rozbudowa grafu o modul ustawien.
  \item Checklist: stabilnosc, restoration, analytics.
\end{itemize}

\section{Elementy do rozbudowy}
\begin{itemize}
  \item Przyklad A: graf nawigacji z parametrami (TODO).
  \item Przyklad B: nested navigation z bottom navigation (TODO).
  \item Zadania: test nawigacji w Compose (TODO).
  \item Pytania: roznica popUpTo i inclusive, jak obslugiwac deep link (TODO).
\end{itemize}

% TODO: uzupelnic narracje, przyklady kodu i cytowania na etapie draftu.

