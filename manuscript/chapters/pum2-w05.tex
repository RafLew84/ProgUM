\chapter{Wyklad 5: StateFlow, SharedFlow i transformacje strumieni}

\begin{learningobjectives}
  \item Rozroznisz StateFlow i SharedFlow w zastosowaniach UI.
  \item Zastosujesz transformacje strumieni w wielowatkowych scenariuszach.
  \item Zaprojektujesz strategie udostepniania stanu w aplikacji. 
\end{learningobjectives}

\section{Mapa rozdzialu}
\begin{itemize}
  \item StateFlow vs LiveData - porownanie i migracja.
  \item SharedFlow, replay, extraBufferCapacity.
  \item stateIn, shareIn, flowOn.
  \item combine, zip, flattenMerge.
  \item Testowanie strumieni hot.
\end{itemize}

\section{Material zrodlowy}
\begin{itemize}
  \item PUM2/Wyk/Wyklad6.pdf.
  \item \texttt{PUM2/Listy/W6-stateIn\_shareIn\_basics.ipynb}, \texttt{PUM2/Listy/W6-flowOn\_basics.ipynb}.
  \item kotlinlang.org/docs/shared-flow.html.
\end{itemize}

\subsection{StateFlow}
\begin{itemize}
  \item Zastosowania w ViewModel i Compose.
  \item UI state, redukcja i initial value.
  \item Konwersja LiveData -> StateFlow.
\end{itemize}

\subsection{SharedFlow}
\begin{itemize}
  \item Konfiguracja replay i buffer.
  \item Event bus, side effects, lost events.
  \item Porownanie z Channel i LiveData.
\end{itemize}

\subsection{Transformacje strumieni}
\begin{itemize}
  \item stateIn, shareIn, flowOn.
  \item combine i zip, flattenMerge i concurrency.
  \item Obsuga backpressure i synchronizacja.
\end{itemize}

\subsection{Case study}
\begin{itemize}
  \item Obsluga eventow UI i state.
  \item Zadanie: implementacja globalnego event bus.
  \item Testowanie z Turbine i runTest.
\end{itemize}

\section{Elementy do rozbudowy}
\begin{itemize}
  \item Przyklad A: stateIn w ViewModel do udostepniania stanu (TODO).
  \item Przyklad B: SharedFlow jako event bus (TODO).
  \item Zadania: porownanie LiveData i StateFlow w projekcie (TODO).
  \item Pytania: kiedy preferowac SharedFlow, jak dobrac replay (TODO).
\end{itemize}

% TODO: uzupelnic narracje, przyklady kodu i cytowania na etapie draftu.
