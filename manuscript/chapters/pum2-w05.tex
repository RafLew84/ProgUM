\chapter{Wyklad 4: Flow i reaktywne zarzadzanie stanem}

\begin{learningobjectives}
  \item Zrozumiesz podstawy przeplywu danych w Kotlin Flow.
  \item Zbudujesz reaktywne pipeline'y dla UI w Compose.
  \item Przygotujesz strategie obslugi bledow i backpressure.
\end{learningobjectives}

\section{Mapa rozdzialu}
\begin{itemize}
  \item Flow cold vs hot - koncepcje podstawowe.
  \item Operatorzy transformacji i kontrola przeplywu.
  \item Buffering, conflation, backpressure.
  \item Integracja z Compose i ViewModel.
  \item Testowanie Flow i deterministyczne scenariusze.
\end{itemize}

\section{Material zrodlowy}
\begin{itemize}
  \item PUM2/Wyk/Wyklad5.pdf.
  \item \texttt{PUM2/Listy/W5-Flow\_basics.ipynb}.
  \item kotlinlang.org/docs/flow.html.
\end{itemize}

\subsection{Podstawy Flow}
\begin{itemize}
  \item Flow builders: flowOf, flow, callbackFlow.
  \item Collect i terminal operators.
  \item Flow vs Sequence - kiedy ktore rozwiazanie.
\end{itemize}

\subsection{Transformacje i operatory}
\begin{itemize}
  \item map, filter, flatMapLatest, combine.
  \item buffer, conflate, distinctUntilChanged.
  \item Error handling w pipeline i retry.
\end{itemize}

\subsection{Integracja z UI}
\begin{itemize}
  \item StateFlow, SharedFlow jako hot streams.
  \item collectAsState i lifecycle aware collect.
  \item Testowanie przeplywu danych w ViewModel.
\end{itemize}

\subsection{Case study}
\begin{itemize}
  \item Przetwarzanie danych katalogu (filtrowanie, sortowanie).
  \item Zadanie: zestaw operatorow do transformacji wynikow API.
  \item Checklist: best practices Flow.
\end{itemize}

\section{Elementy do rozbudowy}
\begin{itemize}
  \item Przyklad A: pipeline filtrowania i sortowania (TODO).
  \item Przyklad B: combine wielu zrodel danych (TODO).
  \item Zadania: testy Flow z Turbine (TODO).
  \item Pytania: roznica cold/hot, kiedy uzyc buffer (TODO).
\end{itemize}

% TODO: uzupelnic narracje, przyklady kodu i cytowania na etapie draftu.
