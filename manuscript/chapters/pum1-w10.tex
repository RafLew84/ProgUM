\chapter{Wyklad 10: Compose + Material Design oraz listy danych}

\begin{learningobjectives}
  \item Wykorzystasz Material Design 3 w Compose do budowy spojnego UI.
  \item Nauczysz sie tworzyc listy i siatki danych z LazyColumn i LazyVerticalGrid.
  \item Zintegrujesz theming, stan UI oraz obsluge danych w interfejsie.
\end{learningobjectives}

\section{Mapa rozdzialu}
\begin{itemize}
  \item Material 3: theming, kolorystyka, typografia.
  \item Komponenty Scaffold, TopAppBar, FloatingActionButton.
  \item LazyColumn, LazyRow, LazyVerticalGrid i zarzadzanie kluczami.
  \item Sposoby ladowania i paginacji danych, placeholdery.
  \item Accessibility i testy UI dla list.
\end{itemize}

\section{Material zrodlowy}
\begin{itemize}
  \item PUM1/Wyk/Wyklad10.pdf.
  \item PUM1/Listy/Lista6.ipynb.
  \item developer.android.com/jetpack/compose/lists.
\end{itemize}

\subsection{Material 3 w Compose}
\begin{itemize}
  \item Theme, color scheme, typography.
  \item Scaffold i sloty layoutu.
  \item Komponenty standardowe (Button, Card, Chip).
\end{itemize}

\subsection{Listy i siatki danych}
\begin{itemize}
  \item LazyColumn, item keys, contentType.
  \item LazyVerticalGrid i optymalizacja.
  \item Sticky headers, sekcje listy i kontrola scroll state.
\end{itemize}

\subsection{Stan i interakcje}
\begin{itemize}
  \item Skeleton loading, error states, empty states.
  \item Pull to refresh, infinite scroll i paging.
  \item Testowanie list: semantics, assertIsDisplayed.
\end{itemize}

\subsection{Case study}
\begin{itemize}
  \item Lista produktow w katalogu.
  \item Integracja z repozytorium danych i paginacja.
  \item Checklist accessibility i theming.
\end{itemize}

\section{Elementy do rozbudowy}
\begin{itemize}
  \item Przyklad A: LazyColumn z placeholderami i diffingiem (TODO).
  \item Przyklad B: Grid produktow z filtrami i paginacja (TODO).
  \item Zadania: zaprojektuj modul listy z obsluga bledow (TODO).
  \item Pytania: roznica miedzy list a lazy list, jak zapewnic stabilne klucze (TODO).
\end{itemize}

% TODO: uzupelnic narracje, przyklady kodu i cytowania na etapie draftu.

