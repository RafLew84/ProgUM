\chapter{Wyklad 6: Obiekty, interfejsy i kompozycja}

\begin{learningobjectives}
  \item Zrozumiesz zastosowania obiektow singleton oraz companion object w Kotlinie.
  \item Wykorzystasz interfejsy i delegacje do budowy modulowego kodu.
  \item Porownasz podejscie kompozycji i dziedziczenia w projektowaniu komponentow.
\end{learningobjectives}

\section{Mapa rozdzialu}
\begin{itemize}
  \item Obiekty w Kotlinie: object declaration, object expression.
  \item Companion object, fabryki i testowalnosc.
  \item Interfejsy: metody domyslne, interfejsy funkcyjne.
  \item Delegacja: by keyword, delegowane wlasciwosci.
  \item Kompozycja jako alternatywa dla dziedziczenia.
\end{itemize}

\section{Material zrodlowy}
\begin{itemize}
  \item PUM1/Wyk/Wyklad6.pdf.
  \item PUM1/Listy/Lista4.ipynb.
  \item kotlinlang.org/docs/object-declarations.html.
\end{itemize}

\subsection{Obiekty i companion}
\begin{itemize}
  \item Singletony i object expression.
  \item Companion object jako fabryka i punkt konfiguracji.
  \item Testowalnosc, zamiana na interfejs i DI.
\end{itemize}

\subsection{Interfejsy}
\begin{itemize}
  \item Domyslne implementacje i dziedziczenie wielokrotne przez interfejsy.
  \item Interfejsy funkcyjne i adaptery SAM.
  \item Rozdzielanie kontraktow na male interfejsy.
\end{itemize}

\subsection{Delegacja i kompozycja}
\begin{itemize}
  \item Delegowane wlasciwosci (lazy, observable, vetoable).
  \item Delegacja implementacji interfejsow przez by.
  \item Kompozycja jako sposob laczenia zachowan.
\end{itemize}

\subsection{Case study}
\begin{itemize}
  \item Projekt modulow uslug w aplikacji.
  \item Analiza zadania: wstrzykiwanie implementacji przez interfejs.
  \item Checklist: unikanie god object i naduzywania singletonow.
\end{itemize}

\section{Elementy do rozbudowy}
\begin{itemize}
  \item Przyklad A: konfiguracja loggera przy pomocy companion object (TODO).
  \item Przyklad B: delegacja odpowiedzialnosci do komponentu cache (TODO).
  \item Zadania: implementacja interfejsu repozytorium z delegacja (TODO).
  \item Pytania: roznice object vs class, zastosowania property delegation (TODO).
\end{itemize}

% TODO: uzupelnic narracje, przyklady kodu i cytowania na etapie draftu.

