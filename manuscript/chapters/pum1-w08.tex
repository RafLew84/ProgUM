\chapter{Wyklad 8: Fundamenty aplikacji Android i cykl zycia}

\begin{learningobjectives}
  \item Przypomnisz cykl zycia kluczowych komponentow Android.
  \item Zaplanujesz zaleznosci pomiedzy Activity, ViewModel i Compose.
  \item Zdefiniujesz strategie reagowania na zmiany cyklu zycia i konfiguracji.
\end{learningobjectives}

\section{Mapa rozdzialu}
\begin{itemize}
  \item Struktura projektu Android (manifest, moduly, zaleznosci).
  \item Activity, Fragment, ViewModel i ich cykl zycia.
  \item LifecycleOwner, observers, SavedStateHandle.
  \item Integracja z Compose: setContent, remember, effect handlers.
  \item Obsuga back stack, rotation, proces-die i recovery.
\end{itemize}

\section{Material zrodlowy}
\begin{itemize}
  \item PUM1/Wyk/Wyklad8.pdf.
  \item PUM1/Listy/Lista5.ipynb.
  \item developer.android.com/guide/components/activities/activity-lifecycle.
\end{itemize}

\subsection{Struktura aplikacji}
\begin{itemize}
  \item Manifest, permissions, konfiguracja gradle.
  \item Rola Application i dependency injection.
  \item Organizacja modulow projektu.
\end{itemize}

\subsection{Lifecycle i ViewModel}
\begin{itemize}
  \item LifecycleOwner, obserwatorzy i eventy.
  \item ViewModel, SavedStateHandle, integracja z repozytoriami.
  \item Compose: remember, LaunchedEffect, DisposableEffect.
\end{itemize}

\subsection{Zmiany konfiguracji}
\begin{itemize}
  \item UI state vs saved instance state.
  \item ViewModel i state hoisting.
  \item Testowanie scenariuszy rotate/background.
\end{itemize}

\subsection{Case study}
\begin{itemize}
  \item Analiza zadania: aktywnosc z Compose i ViewModel.
  \item Lista kontrolna przed releasem (permissions, lifecycle observers).
  \item Porownanie Compose i legacy XML pod katem cyklu zycia.
\end{itemize}

\section{Elementy do rozbudowy}
\begin{itemize}
  \item Przyklad A: rejestrowanie obserwatora lifecycle (TODO).
  \item Przyklad B: implementacja screen state z SavedStateHandle (TODO).
  \item Zadania: symulacja scenariuszy background/foreground (TODO).
  \item Pytania: roznica onPause vs onStop, jak Compose integruje sie z cyklem zycia (TODO).
\end{itemize}

% TODO: uzupelnic narracje, przyklady kodu i cytowania na etapie draftu.

