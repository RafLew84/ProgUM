\chapter{Wyklad 2: Structured concurrency i obsluga wyjatkow}

\begin{learningobjectives}
  \item Zrozumiesz zasady structured concurrency w Kotlinie.
  \item Nauczysz sie propagowac i przechwytywac wyjatki w coroutines.
  \item Zaprojektujesz strategie anulowania i cleanup w asynchronicznych operacjach.
\end{learningobjectives}

\section{Mapa rozdzialu}
\begin{itemize}
  \item Hierarchia scope i kontrola lifecycle.
  \item coroutineScope, supervisorScope, SupervisorJob.
  \item Exception handling i CoroutineExceptionHandler.
  \item Zasady anulowania i cleanup.
  \item Testowanie scenariuszy bledow.
\end{itemize}

\section{Material zrodlowy}
\begin{itemize}
  \item PUM2/Wyk/Wyklad3.pdf.
  \item \texttt{PUM2/Listy/W3-Coroutines\_demo.ipynb}.
  \item kotlinlang.org/docs/exception-handling.html.
\end{itemize}

\subsection{Structured concurrency}
\begin{itemize}
  \item Hierarchia scope i propagacja anulowania.
  \item roznica coroutineScope i supervisorScope.
  \item Granice odpowiedzialnosci modulow.
\end{itemize}

\subsection{Obsluga wyjatkow}
\begin{itemize}
  \item CoroutineExceptionHandler i propagacja bledow.
  \item try/catch w suspend funkcjach.
  \item Retry, exponential backoff, circuit breaker.
\end{itemize}

\subsection{Anulowanie i cleanup}
\begin{itemize}
  \item Cooperative cancellation i sprawdzanie isActive.
  \item finally blok i withContext(NonCancellable).
  \item Timeout i sprzatanie zasobow.
\end{itemize}

\subsection{Case study}
\begin{itemize}
  \item Manager zapytan sieciowych.
  \item Analiza bledow w strumieniach Flow.
  \item Checklist: guidelines obslugi bledow.
\end{itemize}

\section{Elementy do rozbudowy}
\begin{itemize}
  \item Przyklad A: supervisorScope dla wielu zapytan (TODO).
  \item Przyklad B: wzorzec retry dla requestu API (TODO).
  \item Zadania: implementacja cancellation-safe repository (TODO).
  \item Pytania: roznica coroutineScope/supervisorScope, jak dziala handler (TODO).
\end{itemize}

% TODO: uzupelnic narracje, przyklady kodu i cytowania na etapie draftu.
