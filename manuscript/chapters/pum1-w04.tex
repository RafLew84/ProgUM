\chapter{Wyklad 4: Kolekcje i operacje funkcyjne}

\begin{learningobjectives}
  \item Porownasz kolekcje niemutowalne i mutowalne w Kotlinie.
  \item Zastosujesz operacje funkcyjne do przetwarzania danych w aplikacjach mobilnych.
  \item Zaprojektujesz pipeline danych wykorzystujacy sekwencje oraz Flow.
\end{learningobjectives}

\section{Mapa rozdzialu}
\begin{itemize}
  \item Hierarchia kolekcji Kotlin i typy specjalistyczne.
  \item Operatory map, filter, fold oraz sekwencje leniwe.
  \item Kolekcje jako strumienie danych w MVVM i Compose.
  \item Integracja z Flow oraz zrodlami danych offline i online.
  \item Wydajnosc i minimalizacja kopii w strukturach danych.
\end{itemize}

\section{Material zrodlowy}
\begin{itemize}
  \item PUM1/Wyk/Wyklad4.pdf.
  \item PUM1/Listy/Lista2.ipynb.
  \item kotlinlang.org/docs/collections-overview.html.
\end{itemize}

\subsection{Przeglad kolekcji}
\begin{itemize}
  \item List, Set, Map - warianty mutowalne i niemutowalne.
  \item Specjalistyczne implementacje (LinkedHashMap, sorted set).
  \item Kiedy wybrac sequence zamiast kolekcji eager.
\end{itemize}

\subsection{Operacje funkcyjne}
\begin{itemize}
  \item map, filter, flatMap, fold, reduce, groupBy.
  \item Tworzenie lancuchow operacji i unikanie side effects.
  \item Optymalizacja: intermediate vs terminal operations.
\end{itemize}

\subsection{Kolekcje w architekturze}
\begin{itemize}
  \item Przetwarzanie danych w ViewModel przed ekspozycja na UI.
  \item Flow i kolekcje - konwersja pomiedzy typami.
  \item Przekazywanie wynikow do interfejsu Compose.
\end{itemize}

\subsection{Case study}
\begin{itemize}
  \item Zadanie z Listy: agregacja wynikow API.
  \item Porownanie podejsc imperative i declarative.
  \item Testowanie funkcji operujacych na kolekcjach.
\end{itemize}

\section{Elementy do rozbudowy}
\begin{itemize}
  \item Przyklad A: pipeline przetwarzania danych produktow (TODO).
  \item Przyklad B: wykorzystanie sequence do operacji na pliku CSV (TODO).
  \item Zadania: napisz funkcje rozszerzajaca kolekcje z walidacja (TODO).
  \item Pytania: kiedy preferowac lazy kolekcje, jak unikac side effects w lambdach (TODO).
\end{itemize}

% TODO: uzupelnic narracje, przyklady kodu i cytowania na etapie draftu.

