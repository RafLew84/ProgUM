\chapter{Wyklad 11: Czysta architektura i warstwa domeny}

\begin{learningobjectives}
  \item Zdefiniujesz role warstwy domeny w clean architecture.
  \item Zaprojektujesz UseCase i kontrakty interfejsow.
  \item Przygotujesz plan testow jednostkowych dla domeny.
\end{learningobjectives}

\section{Mapa rozdzialu}
\begin{itemize}
  \item Clean architecture rings: domain, data, presentation.
  \item UseCase, entity, repository interface.
  \item Mapowanie DTO -> domain -> UI state.
  \item Error handling i Result wrappers.
  \item Testowanie domeny.
\end{itemize}

\section{Material zrodlowy}
\begin{itemize}
  \item PUM2/Wyk/Wyklad12.pdf.
  \item \texttt{PUM2/Listy/W12-DomainLayer\_basics.ipynb}.
  \item blog.unclebob.com/2012/08/13/the-clean-architecture/.
\end{itemize}

\subsection{Warstwa domeny}
\begin{itemize}
  \item Entities, value objects, domain services.
  \item UseCase jako orkiestracja logiki.
  \item Repository interfaces i granice.
\end{itemize}

\subsection{Mapowanie danych}
\begin{itemize}
  \item DTO -> domain -> ui state.
  \item Result<T> i error mapping.
  \item Walidacja danych i invariants.
\end{itemize}

\subsection{Testowanie}
\begin{itemize}
  \item Testy UseCase.
  \item Fake/mocked repository.
  \item Contract tests dla domeny.
\end{itemize}

\subsection{Case study}
\begin{itemize}
  \item Warstwa domeny katalogu produktow.
  \item Zadanie: UseCase getProductDetails.
  \item Checklist: boundaries i dependencies.
\end{itemize}

\section{Elementy do rozbudowy}
\begin{itemize}
  \item Przyklad A: UseCase z operatorem invoke (TODO).
  \item Przyklad B: Result wrapper i mapowanie bledow (TODO).
  \item Zadania: testy UseCase z fake repository (TODO).
  \item Pytania: kiedy dzielic UseCase, jak modelowac bledy w domenie (TODO).
\end{itemize}

% TODO: uzupelnic narracje, przyklady kodu i cytowania na etapie draftu.
