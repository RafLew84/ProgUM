\chapter{Wyklad 10: Wstrzykiwanie zaleznosci z Hilt}

\begin{learningobjectives}
  \item Poznasz fundamenty Dagger/Hilt w Androidzie.
  \item Zbudujesz modul DI dla warstw MVVM i komponentow Compose.
  \item Zaprojektujesz testowalna konfiguracje zaleznosci.
\end{learningobjectives}

\section{Mapa rozdzialu}
\begin{itemize}
  \item Podstawy DI: komponenty, moduly, scope.
  \item Hilt setup: Application, modules, entry points.
  \item Wstrzykiwanie ViewModel, Repository, UseCase.
  \item Qualifier, provides, binds.
  \item Testowanie z Hilt i zamiana zaleznosci.
\end{itemize}

\section{Material zrodlowy}
\begin{itemize}
  \item PUM2/Wyk/Wyklad11.pdf.
  \item \texttt{PUM2/Listy/W11-DI\_basics.ipynb}.
  \item developer.android.com/training/dependency-injection/hilt-android.
\end{itemize}

\subsection{Wprowadzenie do Hilt}
\begin{itemize}
  \item Konfiguracja i moduly.
  \item Component hierarchy i scope.
  \item EntryPoint i provision w Compose.
\end{itemize}

\subsection{Definiowanie zaleznosci}
\begin{itemize}
  \item @Provides vs @Binds.
  \item Qualifiers i named dependencies.
  \item Injection w ViewModel i w Composables.
\end{itemize}

\subsection{Testowanie}
\begin{itemize}
  \item Hilt test application.
  \item Zamiana bindingow w testach.
  \item Instrumented tests z Hilt.
\end{itemize}

\subsection{Case study}
\begin{itemize}
  \item Moduly DI katalogu produktow.
  \item Zadanie: modul Retrofit/Room/Hilt.
  \item Checklist: scope leaks, initialization.
\end{itemize}

\section{Elementy do rozbudowy}
\begin{itemize}
  \item Przyklad A: modul DI dla repository (TODO).
  \item Przyklad B: test z zamiana zaleznosci (TODO).
  \item Zadania: konfiguracja Hilt w projekcie startowym (TODO).
  \item Pytania: roznica Provides/Binds, jak dziala ViewModelComponent (TODO).
\end{itemize}

% TODO: uzupelnic narracje, przyklady kodu i cytowania na etapie draftu.
