\chapter{Wyklad 6: Kanaly i komunikacja asynchroniczna}

\begin{learningobjectives}
  \item Zrozumiesz role kanalow w Kotlin Coroutines.
  \item Zaprojektujesz komunikacje asynchroniczna pomiedzy komponentami.
  \item Poznasz techniki selekcji i multiplexingu dla strumieni danych.
\end{learningobjectives}

\section{Mapa rozdzialu}
\begin{itemize}
  \item Channel basics: send, receive, close.
  \item Typy kanalow: rendezvous, buffered, conflated.
  \item select i multiplexing.
  \item Porownanie channel vs SharedFlow.
  \item Zastosowania w aplikacjach mobilnych.
\end{itemize}

\section{Material zrodlowy}
\begin{itemize}
  \item PUM2/Wyk/Wyklad7.pdf.
  \item \texttt{PUM2/Listy/W7-Channels\_basics.ipynb}, \texttt{PUM2/Listy/W7-Channels\_select\_basics.ipynb}.
  \item kotlinlang.org/docs/channels.html.
\end{itemize}

\subsection{Podstawy kanalow}
\begin{itemize}
  \item Tworzenie i wykorzystywanie Channel.
  \item send/receive, close, iteration.
  \item Backpressure i buffer.
\end{itemize}

\subsection{Zaawansowane wzorce}
\begin{itemize}
  \item select i multiplexing.
  \item Ticker, pipeline, fan-in i fan-out.
  \item Porownanie z SharedFlow i event bus.
\end{itemize}

\subsection{Praktyka mobilna}
\begin{itemize}
  \item Komunikacja worker -> UI.
  \item Streamowanie danych sieciowych i sensorow.
  \item Testowanie i deterministyczny scheduling.
\end{itemize}

\subsection{Case study}
\begin{itemize}
  \item Kanal powiadomien i logowania.
  \item Analiza bledow: missed signals i resource leaks.
  \item Checklist: czyszczenie zasobow i lifecycle.
\end{itemize}

\section{Elementy do rozbudowy}
\begin{itemize}
  \item Przyklad A: fan-in dla zapytan sieciowych (TODO).
  \item Przyklad B: select dla priorytetyzacji zadan (TODO).
  \item Zadania: testy kanalow z runTest (TODO).
  \item Pytania: kiedy preferowac SharedFlow zamiast Channel (TODO).
\end{itemize}

% TODO: uzupelnic narracje, przyklady kodu i cytowania na etapie draftu.
