\chapter{Wprowadzenie i organizacja (wykład 0)}

\begin{learningobjectives}
  \item Zrozumiesz zasady zaliczenia oraz struktur\k e kursu.
  \item Skonfigurujesz srodowisko Android Studio, emulator oraz narzedzia CLI.
  \item Poznasz plan rozwoju aplikacji przykladowej wykorzystywanej w calym manuskrypcie.
\end{learningobjectives}

\section{Cel rozdzialu}
Ten rozdzial streszcza material ze slajdow `PUM1/Wyk/Wyklad0.pdf` i rozszerza go o praktyczne wskazowki dotyczace pracy w Windows 11 oraz kontroli wersji.

\section{Zarys tresci}
\begin{itemize}
  \item struktura kursow PUM1 i PUM2,
  \item wymagania sprzetowe oraz programowe,
  \item przeglad repozytorium i sposob korzystania z notatnikow `Listy`.
\end{itemize}

\section{Do rozwiniecia}
% TODO: uzupelnic tresc podczas redakcji
\begin{itemize}
  \item Proces instalacji Android Studio (Windows 11) wraz z zrzutami ekranu.
  \item Skracte polecenia CLI przydatne podczas zajec.
  \item Lista checków przed pierwszym laboratorium.
\end{itemize}

